\documentclass[fr]{../../../../../../eplexam}

\hypertitle{Calculabilité}{6}{INGI}{1123}{2018}{Juin}{All}
{Béatrice Desclée \and François Duchêne}
{Yves Deville}

\part{Majeure}
\section{Énoncer et démontrer le théorème de Rice}
\section{Questions vrai/faux avec justification}
\begin{enumerate}
	\item ensemble infini de chaînes de caractères finies est énumérable
	\item sous-ensemble infini d'un ensemble récursif est récursif
	\item toute fonction à domaine fini est calculable
	\item A algébriquement réductible à son complément
	\item un des S-m-n
	\item une machine de Turing avec oracle peut calculer halt(n, x)
	\item NP inclus dans EXPTIME
	\item \ldots
\end{enumerate}
\section{Définir CA, CD, la relation entre les deux et démontrer cette relation}
\section{Définir :}
\begin{enumerate}
	\item La réduction polynomiale
	\item La classe NP
	\item La classe NP-Complet
\end{enumerate}
%%%%%%%%%%%%%%%%%%%%%%%%%%%%%%%%%%%%%%%%%%
\part{Mineure}
\section{Démontrer qu'un programme n'est pas récursif}
Soit A = ($i|Pi$ avec deux outputs distincts et au moins deux inputs distincts)
\begin{enumerate}
	\item Énoncer théorème de Rice
	\item Démontrer pourquoi A non récursif est une conséquence du théorème de Rice
\end{enumerate}
\section{Questions vrai/faux avec justification}
\begin{enumerate}
	\item $N \rightarrow [0,1]$ est non énumérable
	\item Un sous-ensemble infini d'un ensemble récursivement énumérable est énumérable 
	\item \ldots
	\item \ldots
	\item Machine de Turing non-déterministe plus puissante que déterministe
	\item Si $SAT > A$ et si $A \in NP$, alors A est-il NP-complet ?
\end{enumerate}
\section{Définir la réduction algorithmique}
\section{Définir la réduction polynomiale}
\section{Démontrer $A \leq_p B \Rightarrow A \leq_a B$}
\end{document}
